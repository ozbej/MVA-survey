
\chapter{Multidimensional Visual Analysis Tools}

\label{chap:MVATools}

MVA software is a type of computer program that is designed to help
analysts visualize and analyze complex data sets. These programs
typically include a wide range of tools and features that allow users
to create and manipulate graphical representations of the data, such
as scatter plots, parallel coordinates, and heat maps. By using these
tools, analysts can quickly and easily explore the data and gain
insights that might not be immediately apparent from looking at the
raw data. MVA software is commonly used in fields such as business,
finance, and marketing to help make data-driven decisions and uncover
hidden trends in the data. In this section, we will review some
popular MVA software programs.




\section{InfoScope}

InfoScope \parencite{InfoScope} is an interactive visualization tool
to access, explore, and communicate large or complex
datasets. InfoScope is available as free software with the last update
issued on \yearmonthday{2007}{2}{9}. InfoScope is available for Windows, 
macOS, and Linux.

InfoScope can visualize a collection of selected publicly available
datasets, mainly from the finance sector. InfoScope provides an
overview of global relationships between objects by using multiple
views to show different aspects of the data at the same
time. InfoScope provides the following views: \emph{Carto Plot},
\emph{Similarity Map} as high-dimensional projections, \emph{Parallel
Coordinates}, and \emph{Table View}. InfoScope supports brushing and
linking, so all views are highly interactive and tightly
linked. Specific numeric values of attributes can be obtained by
probing the objects in the views, and dynamic queries on a combination
of attributes can be executed using range sliders. All actions are
accompanied by visual feedback within a common frame of reference. The
visual representations make it easy to identify outliers, patterns, or
anomalies.



\section{High-D}

High-D \parencite{HighD} is the successor to InfoScope. It offers
similar functionality with some improvements and added views. High-D
is a versatile tool for revealing hidden features, highlighting trends
and relationships, and finding anomalies in datasets of any size. At
its heart is a powerful interactive parallel coordinates plot for
quick data access, analytical, and presentation purpose. High-D is
available as paid software for 199 USD and with 30-day free evaluation
period. The last update was issued on \yearmonthday{2022}{12}{5}. High-D is
available for Windows, macOS, and Linux.


High-D can visualize a collection of selected publicly available
datasets as well as custom datasets. High-D provides an overview of
global relationships between objects by using multiple views to show
different aspects of the data at the same time. High-D provides the
following views: \emph{Parallel Coordinates}, \emph{Table Plot},
\emph{Distributions}, \emph{Scatter Plot Matrix}, \emph{Parallel
Coordinates Matrix}, \emph{Scatter Plot}, \emph{Similarity Map} as
high dimensional projections (Sammon, Spring, t-SNE, and PCA),
\emph{Tree Map}, and \emph{Carto Plot}. High-D also supports
clustering with k-means++ algorithm. High-D supports brushing and
linking, so all views are highly interactive and tightly linked. It is
possible to obtain specific numeric values of attributes by examining
the objects within the views, and dynamic queries on a combination of
attributes can be performed using range sliders. All actions are
accompanied by visual feedback within a common frame of reference. The
visual representations make it easy to spot unusual data points,
patterns, or anomalies.




\section{GGobi}

GGobi \parencite{cook2007interactive} is a visualization program for
exploring high-dimensional data. GGobi is written in C and is
available as a free and open-source software with the last update
issued on \yearmonthday{2012}{6}{10}. GGobi is available for Windows, Apple macOS,
and Linux.

GGobi can visualize custom datasets and provides a dynamic and
interactive graphics as tours, where data is displayed in an
animation. The data is also available in the following views:
\emph{Scatter Plot}, \emph{Scatter Plot Matrix}, \emph{Parallel
Coordinates}, \emph{Time Series}, \emph{Distributions} as bar
charts. GGobi supports limited brushing. Views offer limited
interactivity and interpretability and are not closely connected.


\section{mVis}

mVis \parencite{CHEGINI20199} is a visual analytics tool for
visualizing multi-dimensional data. mVis is written in Java and is
available as a free and open-source software with the last update
issued on \yearmonthday{2021}{1}{20}. mVis is available for Windows, 
macOS, and Linux.

mVis can visualize custom datasets and provides an overview of global
relationships between objects by using multiple views to show
different aspects of the data at the same time. mVis consists of four
data visualization views: \emph{Scatter Plot Matrix}, \emph{Scatter
Plot}, \emph{Similarity Map} as high dimensional projections (t-SNE,
PCA, and MDS), and \emph{Parallel Coordinates}. mVis also consists of
a panel for controlling data partitions. All of the visualizations are
interconnected through standard brushing and linking, so that any
changes or selections made in one view are reflected in all of the
other views. Additionally, the user has the ability to close,
rearrange, or expand any view as needed.

mVis also supports creating and modifying ML models, specifically
introducing an interactive visual labeling technique that allows an
analyst to build and iteratively improve a ML classification model for
multi-dimensional data sets. This technique combines linked
visualizations, clustering, and active learning to allow the analyst
to interactively label a multi-dimensional dataset in an efficient
manner.



\section{Improvise}

Improvise \parencite{Improvise} is a program that allows users to
create and interact with visualizations that are linked together in
various ways. Improvise is written in Java and is available as a free
and open-source software with the last update issued on 
\yearmonthday{2020}{10}{28}. Improvise is available for Windows, 
macOS, and Linux.

The program uses a shared-object coordination model and a declarative
visual query language to give users control over how data is displayed
in multiple views. This allows users to create visualizations with a
variety of coordination patterns, such as synchronized scrolling,
overview and detail, drill-down, and semantic zoom. Improvise also has
a user interface that allows users to build and explore visualizations
in a live environment, making it easy to modify visualizations as
needed. The goal of Improvise is to provide a high level of
coordination flexibility while also being easy to use. It is designed
to make it simple to create basic coordination patterns and also
possible to create more complex ones.




\section{MyBrush}

MyBrush \parencite{koytek2017mybrush} is an application that allows
users to customize and control the brushing and linking process in
their visualizations. It provides flexibility by allowing users to
specify the source, link, and target of multiple brushes, and supports
a variety of visualization types and multiple simultaneous
brushes. Improvise is written in JavaScript and is available as a free
and open-source web application with the last update issued on
\yearmonthday{2017}{9}{22}.

MyBrush serves as experimental software and offers limited
functionality. Its purpose is to explore the implemented brushing and
linking functionality.  A user can explore a predetermined set of data
with the following views: \emph{Scatter Plot}, \emph{Parallel
Coordinates}, and \emph{Bar Plot}. Any changes or selections made in
one of the visualizations will be reflected in all of the other views
because they are all interconnected through standard brushing and
linking.




\section{XDAT}

XDAT \parencite{XDAT} is a multidimensional data analysis tool
designed to help users quickly and easily extract valuable insights
from large, complex data sets with many variables. XDAT is written in
Java and is available as a free software with the last update issued
on \yearmonthday{2020}{8}{26}. XDAT is available for Windows, 
macOS, and Linux.

XDAT can visualize custom datasets and displays data in separate
views. XDAT displays the data with the following views: \emph{Parallel
Coordinates}, \emph{Table View} and \emph{Scatter Plot}. All of the
visualizations are interconnected through standard brushing and
linking, so that any changes or selections made in one view are
reflected in all of the other views.



\section{TabuVis}

TabuVis \parencite{nguyen2013tabuvis} is a flexible and customizable
visual analytics system that is optimized for analyzing
multidimensional data. Its visualizations can be customized by domain
experts to suit the specific needs of the data being analyzed. TabuVis
is written in Java and is available as a free software with the latest
update issued on \yearmonthday{2022}{2}{19}. TabuVis is available for 
Windows, macOS, and Linux.

TabuVis can visualize custom datasets and displays data in separate
views. TabuVis includes various features for analyzing data, such as
the ability to process data, add automatic marks, create custom
interactive visualizations, and filter the data. These features are
designed to support the entire data analysis process. TabuVis displays
the data in the following views: \emph{Scatter Plots}, \emph{Parallel
Coordinates}, and \emph{Star Plot}. TabuVis doesn't support brushing
and linking.




\section{Comparison of Tools}

A comparison of general information of MVA software can be seen in
Table~\ref{tab:SoftwareGeneral}. A comparison of features of MVA
software can be seen in Table~\ref{tab:SoftwareFeatures}.



\begin{table}[tp]
\tablestretch
\rowcolors{2}{}{tablerowcolour}
\centering
\begin{tabularx}{\linewidth}
{>{\kern-\tabcolsep}lllXl<{\kern-\tabcolsep}}
\toprule
\textbf{Software} & \textbf{Last update} & \textbf{Licence} & \textbf{Systems} & \textbf{Language} \\
\midrule
InfoScope & \yearmonthday{2007}{2}{9} & Free demo & Win, macOS, Linux & ? \\
%
High-D & \yearmonthday{2022}{12}{5} & Commercial & Windows, Apple macOS, and Linux & ? \\
%
GGobi & \yearmonthday{2012}{6}{10} & Free and open-source & Windows, Apple macOS, and Linux & C \\
%
mVis & \yearmonthday{2021}{1}{20} & Free and open-source & Windows, Apple macOS, and Linux & Java \\
%
Improvise & \yearmonthday{2020}{10}{28} & Free and open-source & Windows, Apple macOS, and Linux & Java \\
%
MyBrush & \yearmonthday{2017}{9}{22} & Free and open-source & Web Browser & JavaScript \\
%
XDAT & \yearmonthday{2020}{8}{26} & Free & Windows, Apple macOS, and Linux & Java \\
%
TabuVis & \yearmonthday{2022}{2}{19} & Free & Windows, Apple macOS, and Linux & Java \\
\bottomrule
\end{tabularx}

\caption[Overview of MVA Tools]
{%
Overview of MVA tools.
}
\label{tab:SoftwareGeneral}
\end{table}






\afterpage{%
\clearpage% Flush earlier floats (otherwise order might not be correct)
\thispagestyle{empty}% empty page style (?)
\begin{landscape}% Landscape page
\tablestretch
\rowcolors{2}{}{tablerowcolour}
\centering
\begin{tabularx}{\linewidth}
{>{\kern-\tabcolsep}lcccccccccccccc<{\kern-\tabcolsep}}
\toprule
\thead{Software} & \thead{Custom \\ Datasets} & \thead{Brushing} & \thead{Linking} & \thead{Table \\ View} & \thead{Scatter \\ Plot} & \thead{Scatter \\ Plot \\ Matrix} & \thead{Parallel \\ Coordinates} & \thead{Parallel \\ Coordinates \\ Matrix} & \thead{Similarity \\ Map} & \thead{Time \\ Series} & \thead{Distributions} & \thead{Table \\ Plot} & \thead{Tree \\ Map} & \thead{Carto \\ Plot} \\
\midrule
InfoScope & \checkmark & \checkmark & \checkmark & \checkmark & & & \checkmark & & \checkmark & & & & & \checkmark \\
%
High-D & \checkmark & \checkmark & \checkmark & \checkmark & \checkmark & \checkmark & \checkmark & \checkmark & \checkmark &  & \checkmark & \checkmark & \checkmark & \checkmark \\
%
GGobi & \checkmark & & & & \checkmark & \checkmark & \checkmark & & & \checkmark & \checkmark & & & \\
%
mVis & \checkmark & \checkmark & \checkmark & & \checkmark & \checkmark & \checkmark & & \checkmark & & & & & \\
%
Improvise & \checkmark & & & \checkmark & \checkmark & \checkmark & & & \checkmark & \checkmark & \checkmark & \checkmark & \checkmark & \checkmark \\
%
MyBrush & & \checkmark & \checkmark & & \checkmark & & \checkmark & & & & \checkmark & & & \\
%
XDAT & \checkmark & \checkmark & \checkmark & \checkmark & \checkmark & & \checkmark & & & & & & & \\
%
TabuVis & \checkmark & & & & \checkmark & & \checkmark & & & & & & & \\
\bottomrule
\end{tabularx}
\captionof{table}{Comparison of MVA Tools.}
\label{tab:SoftwareFeatures}
\end{landscape}
\clearpage% Flush page
}


