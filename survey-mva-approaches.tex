\chapter{Multidimensional Visual Analysis Approaches}

\label{chap:MVAApproaches}

MVA approaches are methods and techniques used to analyze and understand complex data sets using
visual representations. These approaches typically involve the use of specialized software or tools that
allow analysts to create and manipulate graphical representations of the data in order to uncover patterns,
trends, and relationships. In this section, we will review some popular MVA approaches.


\section{Brushing and Linking}

Brushing and linking are techniques used in multidimensional visual analysis to allow the user to interact with a visualization and explore data in greater depth.

Brushing refers to the process of selecting data points or regions in one visualization and highlighting those data points in other visualizations. This allows the user to see how the data points or regions of interest are related to other variables in the dataset.

Linking refers to the process of synchronizing the views of multiple visualizations, such that a change made to one visualization is reflected in the other visualizations. This allows the user to explore the data from different perspectives and understand how different variables are related to one another.

Both brushing and linking are useful for helping users to identify patterns and relationships in the data and for facilitating the process of data exploration and analysis.




\section{Scatter Plots}

A scatter plot is a type of graph that is used to display the relationship between two numerical variables, to identify any potential trends or patterns in the data, and to identify outliers. It uses dots or markers to represent the values of the two variables, and position of each dot on the graph indicates the value of the two variables in a single observation.

To create a scatter plot, the values of one variable are plotted on the x-axis (horizontal axis) and the values of the other variable are plotted on the y-axis (vertical axis). The resulting graph will show a set of dots, with each dot representing a single observation. If there is a positive relationship between the two variables, the dots will tend to form a diagonal line that slopes upwards from left to right. If there is a negative relationship, the dots will tend to form a diagonal line that slopes downwards from left to right. If there is no relationship between the two variables, the dots will be scattered randomly across the graph.





\section{High Dimensional Projections}

Visualizing high-dimensional data can be challenging because it is difficult for the human brain to comprehend more than three dimensions. High dimensional projections are techniques that are used to reduce the number of dimensions in the data and represent it in a way that is easier to understand and interpret. High dimensional projections can be further split into \emph{linear} and \emph{non-linear} projections.

\subsection{Principal Component Analysis}

Principal Component Analysis (PCA) is a linear projection. PCA uses linear algebra to identify the underlying dimensions or factors in the data and project the data onto a lower-dimensional space \parencite{abdi2010principal}.


\subsection{Multi-Dimensional Scaling}

Multi-Dimensional Scaling (MDS) is a non-linear projection. MDS uses a distance metric to preserve the distances between data points in the high-dimensional space and project the data onto a lower-dimensional space \parencite{morrison2003fast}.


\subsection{t-Distributed Stochastic Neighbor Embedding}

t-distributed stochastic neighbor embedding (t-SNE) is a non-linear projection. t-SNE uses a probabilistic model to preserve the local structure of the data in the high-dimensional space and project the data onto a lower-dimensional space \parencite{van2008visualizing}.




\section{Parallel Coordinates}

Parallel coordinates are a type of chart that is used to visualize multi-dimensional data. In this type of chart, each data point is represented by a vertical line that extends across multiple axes. The axes are typically arranged in parallel. Each axis represents a different variable, and the position of a point on that axis indicates the value of that variable for the data point. This allows multiple variables to be compared and analyzed simultaneously \parencite{inselberg1990parallel}.

\section{Cluster Analysis}

Cluster analysis is a type of statistical technique used to identify groups of similar objects within a data set. It is a commonly used method for exploratory data analysis, and is often used as a way to gain insight into the underlying structure of the data. In cluster analysis, the data is divided into groups, or clusters, based on the similarity of the objects within each group. This allows analysts to identify common patterns and trends within the data, and to gain a better understanding of the relationships between the objects in the data set \parencite{duran2013cluster}.