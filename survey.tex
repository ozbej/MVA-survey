%----------------------------------------------------------------
%
%  File    :  survey.tex
%
%  Author  :  Keith Andrews, ISDS, TU Graz, Austria
%
%  Created :  24 Mar 2010
%
%  Changed :  22 Jan 2021
%
%----------------------------------------------------------------


\documentclass[11pt,onecolumn,twoside]{report}

\usepackage[
  a4paper,
  twoside,
  top=5mm,                % top margin
  bottom=7mm,             % bottom margin
  inner=20mm,             % inner margin (next to binding)
  outer=20mm,             % outer margin (opposite binding)
  bindingoffset=10mm,     % on binding side
  includeheadfoot,        % include head(er) and foot(er)
  headheight=10mm,        % height of header
  headsep=15mm,           % sep between header and text body
  footskip=15mm,          % sep between body and baseline of footer
  footnotesep = 10mm plus 2mm minus 0mm  % bottom of body to top of footnote
]{geometry}
% A4 paper is w=210m, h=297mm


\newcommand{\thistitle}{Multidimensional Visual Analysis}  % title
\newcommand{\thissubject}{Survey of Multidimensional Visual Analysis Methods and Software}  % subject
% \newcommand{\thissubject}{}  % leave empty for no subject
\newcommand{\thisauthor}{Ožbej Golob}          % author
% \newcommand{\thisauthor}{Keith Andrews, Tom Black, and Harry White}      % multiple authors
\newcommand{\thiskeywords}{multidimensional visual analysis}   % keywords

\newcommand{\thisdate}{11 Jan 2023}  % date of this version
\newcommand{\thisyear}{2023}         % year of this version




\newcommand{\fullh}{24cm}         % height of figures for 1 per page
\newcommand{\halfh}{9.5cm}        % height of figures for 2 per page
\newcommand{\thirdh}{6cm}         % height of figures for 3 per page


\setlength{\parindent}{1em}       % less indentation
\setlength{\parskip}{5pt plus 1pt minus 1pt}  % space before a paragraph


% \tolerance is set by LaTeX to 200
% \sloppy sets \tolerance = 9999
% which allows LaTeX more tolerance in adding word spacing

% \sloppy
% \fussy
% \tolerance = 1000

\tolerance=400 
% makes some lines with lots of white space, but      
% tends to prevent words from sticking out in the margin



\setcounter{tocdepth}{3}        % lowest section level entered in ToC
\setcounter{secnumdepth}{3}     % lowest section level still numbered




\usepackage[T1]{fontenc}        % 8-bit output chars (must be before inputenx)
\usepackage[utf8]{inputenx}     % input char encoding

\usepackage[english,austrian,british]{babel}

\usepackage{newtxtext}          % newer times fonts
\usepackage{newtxmath}

\usepackage{relsize}            % relative font sizes \smaller \larger
\usepackage{float}              % H for float placement
\usepackage{setspace}           % line spacing

\usepackage{textcomp}           % symbols such as \texttimes and \texteuro
\usepackage{latexsym}
\usepackage{fontawesome}        % fontawesome symbols

\usepackage{siunitx}            % prettier number formatting
\sisetup{%
  group-separator={,},
}
\usepackage[super]{nth}         % 1st, 2nd, 3rd, etc.

\usepackage{xspace}
\usepackage{xstring}            % string manipulation macros
\usepackage{xparse}             % commands with optional arguments
\usepackage{etoolbox}           % for \newrobustcmd
\usepackage{makecmds}           % for \makecommand
\usepackage{calc}               % for math calculations

\usepackage[svgnames,table,xcdraw]{xcolor}
\definecolor{darkgreen}{rgb}{0.0,0.2,0.0}
\definecolor{darkblue}{rgb}{0.0,0.0,0.2}
\definecolor{darkred}{rgb}{0.2,0.0,0.0}
\definecolor{verylightgrey}{gray}{0.95}
\definecolor{lightgrey}{gray}{0.9}
\definecolor{grey}{gray}{0.7}
\definecolor{black}{gray}{0.0}

\definecolor{tableheadercolour}{gray}{0.8}
\definecolor{tablerowcolour}{gray}{0.93}



\usepackage{longtable}
\usepackage{multirow}
\usepackage{tabularx}

% Define some new column types for tables:
% like X but flushleft (= raggedright) rather than justified
\newcolumntype{Y}{>{\raggedright\arraybackslash}X}
% a p column but flushleft (= raggedright) rather than justified
\newcolumntype{L}[1]{>{\raggedright\arraybackslash}p{#1}}
% a p column but flushright (= raggedleft) rather than justified
\newcolumntype{R}[1]{>{\raggedleft\arraybackslash}p{#1}}


\usepackage{booktabs}           % nicer tables

\newcommand{\tablestretch}
{\renewcommand{\arraystretch}{1.20}}  % spacing between table rows




\usepackage{verbdef}            % define robust verb strings
\usepackage{verbatim}
\usepackage{comment}



% better lists
\usepackage{enumitem}

\setlist{
  topsep=0pt,
  partopsep=0pt,
  parsep=0.6ex,
  itemsep=1.2ex,
  left=\parindent .. 2\parindent,    % bullet .. start ot text
}

\setlist[description]{
  style=sameline,
}




\usepackage{listings}                 % for listings of source code

\makeatletter
\newlength{\numwidth}%
\setlength{\numwidth}{\widthof{\normalfont{\lst@numberstyle{99}}}}% Up to 2-digit (99) line numbers
\def\lst@PlaceNumber{%
  \makebox[\numwidth+1em][l]{%
    \makebox[\numwidth][r]{\normalfont\lst@numberstyle{\thelstnumber}}%
  }%
}
\makeatother

% lstset strategy: define defaults here for
% all non-floating (displayed) listings
% floated listings override these settings later

\lstset{                              % set parameters for listings
  floatplacement=tp,                  % default float placement
  numberbychapter,
  inputencoding=utf8,
  language=,                          % empty = plain text
  basicstyle=\small\ttfamily,
  tabsize=2,
  xleftmargin=2\parindent,
  xrightmargin=2\parindent,
  frame=none,
  framexleftmargin=0mm,
  rulesepcolor=\color{verylightgrey},
  numbers=none,
  numberstyle=\scriptsize,
  numbersep=2ex,
  breaklines,
  showtabs=false,
  showspaces=false,
  showstringspaces=false,
  keywordstyle=\color{black},
  commentstyle=\color{SteelBlue},
  identifierstyle=,
  stringstyle=,
  captionpos=b,
  abovecaptionskip=\abovecaptionskip,
  belowcaptionskip=\belowcaptionskip,
  extendedchars=true,           % listings usually only support 7-bit ascii chars
  literate=%                    % map some one-byte utf8 chars for use in listings
%    { }{{~}}1                   % non-breaking space
    {©}{{\textcopyright}}1
    {€}{{\texteuro}}1
    {Ö}{{\"O}}1
    {Ä}{{\"A}}1
    {Ü}{{\"U}}1
    {ß}{{\ss}}1
    {ö}{{\"o}}1
    {ä}{{\"a}}1
    {ü}{{\"u}}1,
}


\lstdefinelanguage{biblatex}   % based on biblatex v 2.7a from 2013-07-14
{
  keywords={%
    @article,@book,@mvbook,@inbook,@bookinbook,@suppbook,%
    @booklet,@collection,@mvcollection,@incollection,@suppcollection,%
    @manual,@misc,@online,@patent,@periodical,@suppperiodical,%
    @proceedings,@mvproceedings,@inproceedings,@reference,@mvreference,%
    @inreference,@report,@set,@thesis,@unpublished,@xdata,%
    @conference,@electronic,@mastersthesis,@phdthesis,@techreport,@www,%
    @artwork,@audio,@bibnote,@commentary,@image,@jurisdiction,@legislation,%
    @legal,@letter,@movie,@music,@performance,@review,@software,%
    @standard,@video%
  },
  sensitive=false,
  comment=[l][\itshape]{@comment},
  morecomment=[l]{\%},
}

\lstdefinelanguage{CSS}
{
  alsoletter={-},
  morekeywords={%
  color,background,background-color,margin,padding,font,
  font-family,weight,%
  display,position,top,left,right,bottom,list,%
  style,border,size,white,space,min,width%
  },
  sensitive=false,
  morecomment=[l]{//},
  morecomment=[s]{/*}{*/},
  morestring=[b]",
}





\usepackage[compact,nobottomtitles,pagestyles,explicit]{titlesec}
% when using explicit, must explicitly include #1 for titlename

% nobottomtitles
% move section headings close to page bottom to next page
\renewcommand{\bottomtitlespace}{2cm}

% \chaptermark sets the value of \chaptertitle for later
% \@chapapp is defined as \chaptername outside the appendix,
% and as \appendixname within the appendix.
\makeatletter
\titleformat{\chapter}
[display]                                            % shape
{\chaptermark{\thechapter~~#1}\sffamily\bfseries}    % format
{\huge\@chapapp\ \thechapter}                        % label
{4ex}                                                % sep
{\Huge#1}                                            % before-code
\makeatother

\titleformat{name=\chapter,numberless}
[block]                                              % shape
{\chaptermark{#1}\sffamily\bfseries}                 % format
{}                                                   % label
{0ex}                                                % sep
{\Huge#1}                                            % before-code

\titleformat{\section}
{\normalfont\Large\sffamily\bfseries}{\thesection}{0.8em}{#1}

\titleformat{\subsection}
{\normalfont\large\sffamily\bfseries}{\thesubsection}{0.8em}{#1}

\titleformat{\subsubsection}
{\normalfont\normalsize\sffamily\bfseries}{\thesubsubsection}{0.8em}{#1}

\titleformat{\paragraph}[runin]
{\normalfont\normalsize\sffamily\bfseries}{\theparagraph}{0.8em}{#1}

\titleformat{\subparagraph}[runin]
{\normalfont\normalsize\sffamily\bfseries}{\thesubparagraph}{0.8em}{#1}


% vertical spacing before and after section titles
\titlespacing*{\section}
{0pt}{3.5ex plus 0.5ex minus 0.5ex}{0ex plus 0ex minus 0.2ex}

\titlespacing*{\subsection}
{0pt}{2.5ex plus 0.5ex minus 0.5ex}{0ex plus 0ex minus 0.2ex}

\titlespacing*{\subsubsection}
{0pt}{2ex plus 0.5ex minus 0.5ex}{0ex plus 0ex minus 0.2ex}


% define page headings how I want them

\newpagestyle{main}[\small]{
% \addtolength\headheight{6.7pt}
% \headrule
\sethead%
[{\parbox[t]{0.3\textwidth}%                    % even left
  {\sffamily\thepage}}]
[]%                                             % even centre
[{\parbox[t]{0.6\textwidth}%                    % even right
  {\raggedleft\sffamily\chaptertitle}}]
{{\parbox[t]{0.6\textwidth}%                    % odd left
  {\sffamily\sectiontitle}}}%
{}%                                             % odd centre
{{\parbox[t]{0.3\textwidth}%                    % odd right
  {\raggedleft\sffamily\thepage}}}
}




\usepackage{titletoc}

% \contentsmargin{2.55em}

\titlecontents{chapter}%
[1.5em]%                         % left indent to entry text
{\addvspace{1em}\bfseries}%      % above-code per entry
{\contentslabel{1.5em}}%         % format for numbered entry
{\hspace*{-1.5em}}%              % format for unnumbered entry
{\hfill\contentspage}%           % [no dots] and page num per entry


% Note: \dottedcontents is short form of \titlecontents

\dottedcontents{section}%
[3.8em]%                         % left indent to entry text = 1.5 + 2.3
{}%                              % above-code per entry
{2.3em}%                         % label width
{1pc}%                           % space around the dots

\dottedcontents{subsection}%
[7.4em]%                         % left indent to entry text = 3.8 + 3.6
{}%                              % above-code per entry
{3.6em}%                         % label width
{1pc}%                           % space around the dots


\dottedcontents{figure}%         % LoF entries
[3.0em]%                         % left indent to entry text = 3.8 + 3.6
{}%                              % above-code per entry
{3.0em}%                         % label width
{1pc}%                           % space around the dots

\dottedcontents{table}%          % LoT entries
[3.0em]%                         % left indent to entry text = 3.8 + 3.6
{}%                              % above-code per entry
{3.0em}%                         % label width
{1pc}%                           % space around the dots



% List of Listings is unknown to titletoc, define here

% Add extra per-chapter space to LoL to mimic LoF and LoT
% (requires package etoolbox)
\makeatletter
\patchcmd{\@chapter}% <cmd>
  {\addtocontents}% <search>
  {\addtocontents{lol}{\protect\addvspace{10\p@}}% add per-chapter space
   \addtocontents}% <replace>
  {}{}% <success><failure>
\makeatother

% Configure LoL to mimic LoF and LoT
\contentsuse{lstlisting}{lol}

\titlecontents{lstlisting}%
[3.0em]%                              % left indent
{\addvspace{1.5mm}}%                  % above-code per entry
{\contentslabel{3.0em}}%              % format for numbered entry
{\hspace*{-3.0em}}%                   % format for unnumbered entry
{\titlerule*[1pc]{.} \contentspage}%  % dots and page num per entry
[]%                                   % below-code per entry

\renewcommand{\lstlistlistingname}{List of Listings}






% sensible settings for floats

\setlength{\textfloatsep}{9mm plus 2mm minus 2mm}
\setlength{\floatsep}{9mm plus 2mm minus 2mm}
\setlength{\intextsep}{9mm plus 2mm minus 2mm}

\setlength{\dbltextfloatsep}{9mm plus 2mm minus 2mm}
\setlength{\dblfloatsep}{9mm plus 2mm minus 2mm}

\setlength{\abovecaptionskip}{4mm plus 2mm minus 1mm}
\setlength{\belowcaptionskip}{2mm plus 1mm minus 1mm}


% See http://www-rohan.sdsu.edu/~aty/bibliog/latex/floats.html
% See https://robjhyndman.com/hyndsight/latex-floats/

\setcounter{topnumber}{2}               % max num floats at top of page
\setcounter{dbltopnumber}{2}            % max num floats on 2col page
\setcounter{bottomnumber}{2}            % max num floats at bottom of page
\setcounter{totalnumber}{4}             % max num floats on a page

\renewcommand{\topfraction}{0.8}        % max fraction of floats at top
\renewcommand{\dbltopfraction}{0.9}     % max fraction of floats at top 2col
\renewcommand{\bottomfraction}{0.8}     % max fraction of floats at bottom
\renewcommand{\textfraction}{0.2}       % min fraction of text

% only for entirely float pages:
\renewcommand{\floatpagefraction}{0.7}      % min page fraction having floats
\renewcommand{\dblfloatpagefraction}{0.7}   % min 2col page fraction having floats


% \usepackage[section,above,below]{placeins}  % keep floats to their own section




% use caption and subfig (caption2 and subfigure are now obsolete)

\usepackage[
  position=bottom,
  margin=1cm,
  font=small,
  labelfont={bf,sf},
  format=plain,
  indention=5mm,
  aboveskip=4mm,
  belowskip=0mm,
]{caption,subfig}

\captionsetup[subfigure]{
  margin=0pt,
  parskip=0pt,
  indention=5mm,
  farskip=4mm,            % skip above subfig (assuming captions at bottom)
  captionskip=2mm,        % skip between subfig and subcaption
}




\usepackage[short]{datetime}   % load datetime *after* babel, requires fmtcount
% \newdateformat{britdate}{%
% \ordinaldate{\THEDAY} \,\monthname[\THEMONTH] \THEYEAR
% }
\newdateformat{unixdate}{%
\twodigit{\THEDAY}~\shortmonthname[\THEMONTH]~\THEYEAR
}



\usepackage[
  autostyle=true,          % adapt quote style to current language
  english=british,         % british english as default
  threshold=1,             % set block quotations >1 line in display mode
  maxlevel=4,              % max nesting level
]{csquotes}

\usepackage[
  indentfirst=false,
  vskip=0pt,               % by default would be \topsep + \partopsep.
]{quoting}

% tell csquotes to use quoting environment
% for \displayquote and \blockquote
\SetBlockEnvironment{quoting}

% if cite is issued by a csquote command
\renewcommand{\mkcitation}[1]{\space#1}

% I prefer double quotes as outer
\DeclareQuoteStyle{keithbritish}%  [variant]{style}
  {\textquotedblleft}%                      opening outer mark
  {\textquotedblright}%                     closing outer mark
  [0.05em]%
  {\textquoteleft}%                         opening inner mark
  {\textquoteright}%                        closing inner mark

\ExecuteQuoteOptions{style=keithbritish}





\usepackage[
  backend=biber,
%  style=ext-authoryear-comp,   % defined in biblatex-ext package
  style=ext-authoryear,        % defined in biblatex-ext package
  sorting=nyt,
  useprefix,                   % van and von are part of second name
  mergedate=false,             % only for authoryear style
  dashed=false,                % only for authoryear style
  abbreviate=false,
  maxcitenames=2,              % if > 2 authors,
  mincitenames=1,              % use first 1 then et al
  maxbibnames=99,              % if > 99 authors,
  minbibnames=6,               % use first 6 then et al
  uniquelist=minyear,
  uniquename=init,
  hyperref=true,
  backref=true,
  backrefstyle=two,
  sortlocale=en,
]{biblatex}



% set for csquotes, but \autocite only available
% after biblatex is loaded
\SetCiteCommand{\autocite}    % or maybe \parencite

% more space between entries in bib
\setlength\bibitemsep{1.5\itemsep}

% kandrews: replace round brackets with square brackets in citations
\DeclareOuterCiteDelims{parencite}{\bibopenbracket}{\bibclosebracket}
\DeclareInnerCiteDelims{textcite}{\bibopenbracket}{\bibclosebracket}

% kandrews: replace round brackets with square brackets in bibliography
% biblabeldate is a biblatex-ext feature
\DeclareFieldFormat{biblabeldate}{\mkbibbrackets{#1}}


% remove URL: from in front of URLs
\DeclareFieldFormat{url}{\url{#1}}
\DeclareFieldFormat{doi}{\doi{#1}}
\DeclareFieldFormat{isbn}{\isbn{#1}}
\DeclareFieldFormat{issn}{\issn{#1}}

% suppress urldate field
\AtEveryBibitem{\clearfield{urlyear}}

% remove In: from @article and @inproceedings entries
% https://tex.stackexchange.com/questions/10682/suppress-in-biblatex
\renewbibmacro{in:}{%
  \ifboolexpr{%
     test {\ifentrytype{article}}%
     or
     test {\ifentrytype{inproceedings}}%
  }{}{\printtext{\bibstring{in}\intitlepunct}}%
}

% make all entry titles italic
% (also removes quotation marks from around titles)
% https://tex.stackexchange.com/questions/311816/want-title-in-simple-numeric-not-italic-through-bibliography
\DeclareFieldFormat*{title}{\mkbibitalic{#1}}
\DeclareFieldFormat*{citetitle}{\mkbibitalic{#1}}

% make journal names non-italic
\DeclareFieldFormat{journaltitle}{#1\isdot}

% make proceedings names non-italic
\DeclareFieldFormat[inproceedings]{booktitle}{#1\isdot}

% use nth for edition
\DeclareFieldFormat{edition}{%
  \ifinteger{#1}
    {\nth{#1}~\bibstring{edition}}
    {#1\isdot}}

% overwrite some standard strings in english.lbx
\DefineBibliographyStrings{english}{%
  edition          = {Edition},
  mathesis         = {Master's Thesis},
  phdthesis        = {PhD\addabbrvspace Thesis},
}


% kandrews
% use Unix format for dates in biblio:
% 29 Dec 2015, 01 Oct 2018, etc.

% for now, define under lang english not british
% due to bug in biblatex 3.11

\DefineBibliographyStrings{english}{%
  january          = {Jan},
  february         = {Feb},
  march            = {Mar},
  april            = {Apr},
  may              = {May},
  june             = {Jun},
  july             = {Jul},
  august           = {Aug},
  september        = {Sep},
  october          = {Oct},
  november         = {Nov},
  december         = {Dec},
}

\DefineBibliographyExtras{english}{%
% #1 = year, #2 = month, #3 = day
\protected\def\mkbibdatelong#1#2#3{%
  \iffieldundef{#3}
    {}
    {\mkdayzeros{\thefield{#3}}%
     \iffieldundef{#2}{}{\nobreakspace}}%
  \iffieldundef{#2}
    {}
    {\mkbibmonth{\thefield{#2}}%
     \iffieldundef{#1}{}{\space}}%
  \iffieldbibstring{#1}{\bibstring{\thefield{#1}}}{\mkyearzeros{\thefield{#1}}}}%
%
\protected\def\mkbibdateshort#1#2#3{%
  \iffieldundef{#3}
    {}
    {\mkdayzeros{\thefield{#3}}%
     \iffieldundef{#2}{}{\nobreakspace}}%
  \iffieldundef{#2}
    {}
    {\mkbibmonth{\thefield{#2}}%
     \iffieldundef{#1}{}{\space}}%
  \iffieldbibstring{#1}{\bibstring{\thefield{#1}}}{\mkyearzeros{\thefield{#1}}}}%
}



\addbibresource{mva.bib}




% xurl provides better URL breaking than url
% load after biblatex
\usepackage[hyphens,obeyspaces]{xurl}
\def\UrlFont{\smaller\ttfamily}






\usepackage{ifpdf}

\ifpdf
  % pdflatex
  \usepackage[pdftex]{graphicx}
  \DeclareGraphicsExtensions{.pdf,.jpg,.png}
  \pdfcompresslevel=9
  \pdfobjcompresslevel=1  % also compress PDF object streams except embedded PDFs
  \pdfpageheight=297mm
  \pdfpagewidth=210mm
  \usepackage[         % hyperref should be last package loaded
    unicode,
    pdftex,
    pdfversion=1.7,
    pdftitle={\thistitle},
    pdfsubject={\thissubject},
    pdfauthor={\thisauthor},
    pdfkeywords={\thiskeywords},
    bookmarks,
    bookmarksnumbered,
    linktocpage,
    colorlinks,
    linkcolor=darkred,
    anchorcolor=red,
    citecolor=darkgreen,
    urlcolor=darkblue,
    pdfstartview=Fit,              % initial view
    pdfview=Fit,                   % view after following a link
    pdfpagelayout=SinglePage,      % single page, no scrolling
    pdfpagemode=UseOutlines,       % open bookmarks in Acrobat
    plainpages=false,              % avoids duplicate page number problem
    pdfpagelabels,                 % avoids duplicate page number problem
    breaklinks=true,               % allow links exceeding a single line
  ]{hyperref}

\else
  % latex
  \usepackage[dvips]{graphicx}
  \DeclareGraphicsExtensions{.eps}
  \usepackage[dvips]{hyperref}
\fi


% export adjustbox keys to includegraphics
% must be after \usepackage{graphicx}
\usepackage[export]{adjustbox}    % valign=t, frame, ...

\usepackage{pdflscape}
\usepackage{afterpage}
\usepackage{capt-of}
\usepackage{amsmath}
\usepackage{makecell}

\renewcommand\theadalign{bc}
\renewcommand\theadfont{\bfseries}
\renewcommand\theadgape{\Gape[4pt]}
\renewcommand\cellgape{\Gape[4pt]}



% subset of macros from thesis-macros

% \liintro list item intro is a style used when list items have an
% introduction phrase (say in italics) followed by a colon.
\newcommand{\liintro}[1]{\emph{#1}}

% short notes in square brackets
\newcommand{\shortnote}[1]
{%
{{\smaller{}[#1]}}
}


\newcommand{\TODO}[1]
{
{\textcolor{red}{[TODO: #1]}}
}



\newcommand{\imgcredit}[1]
{\smaller{}[#1]}



\newcommand{\copyrightACM}
{%
Copyright \copyright\ by the Association for Computing Machinery, Inc.%
}




\newcommand{\daymonthyear}[3]
{%
\twodigit{#1}\hspace{0.7ex}\nolinebreak[2]\shortmonthname[#2]\hspace{0.7ex}\nolinebreak[2]#3%
}


\newcommand{\monthyear}[2]
{%
\shortmonthname[#1]\hspace{0.7ex}\nolinebreak[2]#2%
}


\newcommand{\yearmonthday}[3]
{%
\twodigit{#3}\hspace{0.7ex}\nolinebreak[2]\shortmonthname[#2]\hspace{0.7ex}\nolinebreak[2]#1%
}


\newcommand{\yearmonth}[2]
{%
\shortmonthname[#2]\hspace{0.7ex}\nolinebreak[2]#1%
}



% link to Amazon or
% http://worldcatlibraries.org/wcpa/isbn/[ISBN number]
% http://amazon.com/exec/obidos/ASIN/#1/keithandrewshcic
% http://amazon.com/dp/#1/

\newrobustcmd{\isbn}[1]
{%
{%
\ifpdf
{\smaller ISBN
\href{http://amazon.co.uk/dp/#1/}{#1}}%
\else
{\smaller ISBN #1}%
\fi
}%
}



% ISSN
% http://www.bl.uk/services/bibliographic/issn.html
% 8 digits, should be printed xxxx-xxxx
% e.g. 0020-0190 is Information Processing Letters, Elsevier
%
% Lookup services:
% http://kmittlib.lib.kmutt.ac.th:81/search/i?SEARCH=0020-0190
% http://worldcatlibraries.org/wcpa/issn/0020-0190

\newrobustcmd{\issn}[1]
{%
{%
\ifpdf
{\smaller ISSN
\href{http://worldcatlibraries.org/wcpa/issn/#1}{#1}}%
\else
{\smaller ISSN #1}%
\fi
}%
}



% DOIs  http://doi.org/  e.g.
% doi:10.1038/nature723
% http://doi.org/10.1038/nature723

\newrobustcmd{\doi}[1]
{%
{%
\def\UrlFont{\smaller\rmfamily}
\ifpdf                                   % pdflatex
\href{http://doi.org/#1}{doi:\protect\nolinkurl{#1}}%
\else                                    % latex
doi:\protect\nolinkurl{#1}%
\fi
}%
}





\newrobustcmd{\website}[1]
{%
\ifpdf                                  % pdflatex
\href{http://#1/}{\nolinkurl{#1}}%
\else                                   % latex
\nolinkurl{#1}%
\fi
}




\newcommand{\news}[1]
{%
\ifpdf
\href{news:#1}{\nolinkurl{#1}}
\else
\nolinkurl{#1}%
\fi
}








% based on url package
% define styles for class, file, and variable names
% which break nicely at line breaks

% make the macros robust so they work inside captions, etc

\newcommand{\ttname}{\begingroup \smaller\urlstyle{tt}\Url}
\newcommand{\rmname}{\begingroup \smaller\urlstyle{rm}\Url}
\newcommand{\sfname}{\begingroup \smaller\urlstyle{sf}\Url}


% fname is for file names and directory names
\newrobustcmd{\fname}[1]{\ttname{#1}}

% vname is for variable names, domain names, email addresses
\newrobustcmd{\vname}[1]{\ttname{#1}}




% for class names, define our own url style

\makeatletter  % protect @ names

% \url@letstyle: New URL style to premit break at any letters.
% Based on \url@ttstyle

\def\Url@letdo{% style assignments for tt fonts or T1 encoding
\def\UrlBreaks{\do\a\do\b\do\c\do\d\do\e\do\f\do\g\do\h\do\i\do\j\do\k\do\l%
               \do\m\do\n\do\o\do\p\do\q\do\r\do\s\do\t\do\u\do\v\do\w\do\x%
               \do\y\do\z%
               \do\A\do\B\do\C\do\D\do\E\do\F\do\G\do\H\do\I\do\J\do\K\do\L%
               \do\M\do\N\do\O\do\P\do\Q\do\R\do\S\do\T\do\U\do\V\do\W\do\X%
               \do\Y\do\Z%
}%
\def\UrlBigBreaks{\do\.\do\@\do\\\do\/\do\!\do\_\do\|\do\%\do\;\do\>\do\]%
 \do\)\do\,\do\?\do\'\do\+\do\=\do\#\do\:\do@url@hyp}%
\def\UrlNoBreaks{\do\(\do\[\do\{\do\<}% (unnecessary)
\def\UrlSpecials{\do\ {\ }}%
\def\UrlOrds{\do\*\do\-\do\~}% any ordinary characters that aren't usually
\Urlmuskip = 0mu plus 1mu%
}

\def\url@letstyle{%
\@ifundefined{selectfont}{\def\UrlFont{\sf}}{\def\UrlFont{\sffamily}}\Url@letdo
}

\makeatother  % unprotect @ names

% class names
\newcommand\letname{\begingroup \smaller\urlstyle{let}\Url}

\newrobustcmd{\cname}[1]{\letname{#1}}


% ui element names
\newrobustcmd{\uiname}[1]{{\smaller\textsf{#1}}}

% html5 element names
\newrobustcmd{\elname}[1]{{\lstinline{<#1>}}}

% css class names
\newrobustcmd{\cssname}[1]{{\lstinline{#1}}}



% Euro symbol
\newcommand{\euro}{\texteuro\,}

% times symbol
\newcommand{\timessym}{\texttimes\,}

% approx symbol
\newcommand{\approxsym}{\ensuremath\approx\,}

% plusminus symbol
\newcommand{\plusminussym}{\textpm\,}

% not equal symbol
\newcommand{\neqsym}{\ensuremath{\neq\,}}

% rightarrow symbol
\newcommand{\rightarrowsym}{\ensuremath\rightarrow\,\,}


% thumbs up and thumbs down symbols

\newcommand{\uthumb}{\smaller[2]\raisebox{1pt}{\textcolor{DarkGreen}{\faThumbsUp}}}

\newcommand{\dthumb}{\smaller[2]\raisebox{1pt}{\textcolor{DarkRed}{\faThumbsDown}}}







\begin{document}

\unixdate

\normalsize
\pagestyle{empty}         % for preliminary pages (no numbers shown)
\pagenumbering{Roman}     % for pdf labels




\begin{titlepage}

\begin{center}

\begin{spacing}{1.1}
\Large\sffamily\bfseries
\thistitle
\end{spacing}

\ifstrempty{\thissubject}{}%     % if empty subject string, do nothing
{%
\begin{spacing}{1.1}
\large\sffamily\bfseries
\thissubject
\end{spacing}
}


\vspace{1cm}

{\large\sffamily \thisauthor}

% {\large\sffamily Group 4}
% \vspace{5mm}
% {\large\sffamily Keith Andrews, Tom Strong, Bill Weak, and Seb Green}

\vspace{1cm}

% Institute of Interactive Systems and Data Science (ISDS), \\
% Graz University of Technology \\
% A-8010 Graz, Austria \\[1cm]


{\large
% 706.041 Information Architecture and Web Usability 3VU WS 2021/2022 \\
Graz University of Technology \\[1cm]
}


\vspace{1cm}

\thisdate

\end{center}



\vspace{2cm}

\begin{quote}
\begin{center}
{\large\sffamily\bfseries Abstract}
\end{center}
Writing a survey can be a traumatic endeavour. It might be a student's
first foray into academic research. There are often obstacles and
false dawns along the way. This survey paper takes a fresh look at the
process and addresses new ways of accomplishing this daunting goal.

The abstract should concisely describe what the survey is about.
State the areas which are covered and also those which are not
covered. Market your survey to your readership. Also, make sure you
mention all relevant keywords in the abstract, since many readers read
\emph{only} the abstract and many search engines index \emph{only} the
title and the abstract.

This survey explores the issues concerning the writing of an academic
survey paper and presents numerous novel insights. Special attention
is paid to the use of clear and simple English for an international
audience, and advice is given regarding the use of technical aids to
production.
\end{quote}

\vfill

\begin{center}
{\footnotesize\sffamily \copyright~Copyright \thisyear{} by the author(s),
except as otherwise noted.}

\vspace{2mm}
{\footnotesize\sffamily This work is placed under a
Creative Commons Attribution 4.0 International
(\href{https://creativecommons.org/licenses/by/4.0/}{CC BY 4.0}) licence.}
\end{center}

\end{titlepage}




\cleardoublepage
\pagestyle{plain}             % for preliminary pages
\pagenumbering{roman}         % for preliminary pages


\begin{spacing}{0.8}
\tableofcontents
\end{spacing}
\addcontentsline{toc}{chapter}{Contents}

\cleardoublepage
\begin{spacing}{0.8}
\listoffigures
\end{spacing}
\addcontentsline{toc}{chapter}{List of Figures}

\cleardoublepage
\begin{spacing}{0.8}
\listoftables
\end{spacing}
\addcontentsline{toc}{chapter}{List of Tables}

\cleardoublepage
\begin{spacing}{0.8}
\renewcommand{\lstlistlistingname}{List of Listings}
\lstlistoflistings
\end{spacing}
\addcontentsline{toc}{chapter}{List of Listings}



\cleardoublepage
\pagestyle{main}              % for main pages
\pagenumbering{arabic}        % for main pages


\cleardoublepage
%----------------------------------------------------------------
%
%  File    :  survey-intro.tex
%
%  Author  :  Keith Andrews, IICM, TU Graz, Austria
% 
%  Created :  27 May 1993
% 
%  Changed :  16 Nov 2010
% 
%----------------------------------------------------------------


\chapter{Introduction}

\label{chap:Intro}



Multidimensional Visual Analysis (MVA) is a field that focuses on the
use of visual representations to explore and analyze complex data
sets. This approach aims to provide a more intuitive and interactive
way to understand and extract insights from large and complex
datasets.

One of the main challenges in MVA is the effective representation of
high-dimensional data in a way that is meaningful and intuitive for
the user. To address this challenge, a wide range of visual encodings
and interaction techniques have been developed, including scatter
plots, parallel coordinates, and multidimensional scaling.

One of the key benefits of MVA is its ability to reveal patterns and
relationships in the data that may not be apparent through traditional
statistical analysis methods. This is particularly useful in the
exploratory phase of data analysis, where the aim is to gain a better
understanding of the data and identify potential areas of interest for
further investigation.

In addition to its use in exploratory data analysis, MVA also has a
range of applications in areas such as data mining, machine learning,
and business intelligence. By enabling users to interact with and
visualize data in a more intuitive way, it can help to improve
decision-making and facilitate the discovery of insights and trends
that may not have been identified through other means.

Overall, MVA is an important tool for understanding and making sense
of complex data, and has the potential to greatly enhance our ability
to extract knowledge and insights from large and complex datasets.

In this survey, we will review multiple popular MVA approaches and
popular MVA software. See Chapter~\ref{chap:MVA} for MVA
approaches and Chapter~\ref{chap:MVATools} for MVA software.



\cleardoublepage

\chapter{Multidimensional Visual Analysis (MVA)}

\label{chap:MVA}

Multidimensional Visual Analysis (MVA) approaches are methods and
techniques used to analyze and understand complex data sets using
visual representations. These approaches typically involve the use of
specialized software or tools that allow analysts to create and
manipulate graphical representations of the data in order to uncover
patterns, trends, and relationships. In this section, we will review
some popular MVA approaches.



\section{Scatter Plots}

A scatter plot is a type of graph that is used to display the
relationship between two numerical variables, to identify any
potential trends or patterns in the data, and to identify outliers. It
uses dots or markers to represent the values of the two variables, and
position of each dot on the graph indicates the value of the two
variables in a single observation.

To create a scatter plot, the values of one variable are plotted on
the x-axis (horizontal axis) and the values of the other variable are
plotted on the y-axis (vertical axis). The resulting graph will show a
set of dots, with each dot representing a single observation. If there
is a positive relationship between the two variables, the dots will
tend to form a diagonal line that slopes upwards from left to
right. If there is a negative relationship, the dots will tend to form
a diagonal line that slopes downwards from left to right. If there is
no relationship between the two variables, the dots will be scattered
randomly across the graph.





\section{Similarity Maps}

Similarity maps are projections of high-dimensional datasets to two
(or sometimes three) dimensions. Projection techniques are used to
reduce the number of dimensions in the data, while attempting to
preserve distances between items as far as possible. Items which are
close in the high-dimensional space should also be close in the
resulting two-dimensional projection space. Such projections can be
further split into \emph{linear} and \emph{non-linear} projections.


\subsection{Principal Component Analysis (PCA)}

Principal Component Analysis (PCA) is a linear projection. PCA uses
linear algebra to identify the underlying dimensions or factors in the
data and project the data onto a lower-dimensional space
\parencite{abdi2010principal}.


\subsection{Multi-Dimensional Scaling (MDS)}

Multi-Dimensional Scaling (MDS) is a non-linear projection. MDS uses a
distance metric to preserve the distances between data points in the
high-dimensional space and project the data onto a lower-dimensional
space \parencite{morrison2003fast}.


\subsection{t-Distributed Stochastic Neighbor Embedding (t-SNE)}

t-distributed stochastic neighbor embedding (t-SNE) is a non-linear
projection. t-SNE uses a probabilistic model to preserve the local
structure of the data in the high-dimensional space and project the
data onto a lower-dimensional space \parencite{van2008visualizing}.


% UMAP





\section{Parallel Coordinates}

Parallel coordinates are a type of chart that is used to visualize
multi-dimensional data. In this type of chart, each data point is
represented by a vertical line that extends across multiple axes. The axes
are typically arranged in parallel. Each axis represents a different
variable, and the position of a point on that axis indicates the value of
that variable for the data point. This allows multiple variables to be
compared and analyzed simultaneously \parencite{inselberg1990parallel}.







\section{Brushing and Linking}

Brushing and linking are techniques used in multidimensional visual
analysis to allow the user to interact with a visualization and
explore data in greater depth.

Brushing refers to the process of selecting data points or regions in
one visualization and highlighting those data points in other
visualizations. This allows the user to see how the data points or
regions of interest are related to other variables in the dataset.

Linking refers to the process of synchronizing the views of multiple
visualizations, such that a change made to one visualization is
reflected in the other visualizations. This allows the user to explore
the data from different perspectives and understand how different
variables are related to one another.

Both brushing and linking are useful for helping users to identify
patterns and relationships in the data and for facilitating the
process of data exploration and analysis.






\section{Grouping and Labelling}

Grouping and labeling data is a fundamental step in the ML process.
Grouping refers to the process of identifying and separating similar data
points, while labeling refers to the process of providing relevant
information about the data. Grouping and labelling data helps to organize
and structure the data, making it easier to understand and work with. It
allows for more accurate and meaningful analysis of the data, as similar
data points can be grouped together and analyzed in relation to one
another. It also allows the ML model to understand the context of the data
and make more accurate predictions, and enables the interpretability of
the model, by providing meaningful information about the data and how it
is being used.

\subsection{Manual Grouping}

Manual grouping and labeling refers to the process of organizing and
providing relevant information about data manually, typically done by a
human. The process of manual grouping and labeling can be time-consuming
and labor-intensive.




\subsection{Automated Clustering}

Cluster analysis is a type of statistical technique used to identify
groups of similar objects within a data set. It is a commonly used
method for exploratory data analysis, and is often used as a way to
gain insight into the underlying structure of the data. In cluster
analysis, the data is divided into groups, or clusters, based on the
similarity of the objects within each group. This allows analysts to
identify common patterns and trends within the data, and to gain a
better understanding of the relationships between the objects in the
data set \parencite{duran2013cluster}.





\cleardoublepage

\chapter{Multidimensional Visual Analysis Tools}

\label{chap:MVATools}

MVA software is a type of computer program that is designed to help
analysts visualize and analyze complex data sets. These programs
typically include a wide range of tools and features that allow users
to create and manipulate graphical representations of the data, such
as scatter plots, parallel coordinates, and heat maps. By using these
tools, analysts can quickly and easily explore the data and gain
insights that might not be immediately apparent from looking at the
raw data. MVA software is commonly used in fields such as business,
finance, and marketing to help make data-driven decisions and uncover
hidden trends in the data. In this section, we will review some
popular MVA software programs.




\section{InfoScope}

InfoScope \parencite{InfoScope} is an interactive visualization tool
to access, explore, and communicate large or complex
datasets. InfoScope is available as free software with the last update
issued on \yearmonthday{2007}{2}{9}. InfoScope is available for Windows, 
macOS, and Linux.

InfoScope can visualize a collection of selected publicly available
datasets, mainly from the finance sector. InfoScope provides an
overview of global relationships between objects by using multiple
views to show different aspects of the data at the same
time. InfoScope provides the following views: \emph{Carto Plot},
\emph{Similarity Map} as high-dimensional projections, \emph{Parallel
Coordinates}, and \emph{Table View}. InfoScope supports brushing and
linking, so all views are highly interactive and tightly
linked. Specific numeric values of attributes can be obtained by
probing the objects in the views, and dynamic queries on a combination
of attributes can be executed using range sliders. All actions are
accompanied by visual feedback within a common frame of reference. The
visual representations make it easy to identify outliers, patterns, or
anomalies.



\section{High-D}

High-D \parencite{HighD} is the successor to InfoScope. It offers
similar functionality with some improvements and added views. High-D
is a versatile tool for revealing hidden features, highlighting trends
and relationships, and finding anomalies in datasets of any size. At
its heart is a powerful interactive parallel coordinates plot for
quick data access, analytical, and presentation purpose. High-D is
available as paid software for 199 USD and with 30-day free evaluation
period. The last update was issued on \yearmonthday{2022}{12}{5}. High-D is
available for Windows, macOS, and Linux.


High-D can visualize a collection of selected publicly available
datasets as well as custom datasets. High-D provides an overview of
global relationships between objects by using multiple views to show
different aspects of the data at the same time. High-D provides the
following views: \emph{Parallel Coordinates}, \emph{Table Plot},
\emph{Distributions}, \emph{Scatter Plot Matrix}, \emph{Parallel
Coordinates Matrix}, \emph{Scatter Plot}, \emph{Similarity Map} as
high dimensional projections (Sammon, Spring, t-SNE, and PCA),
\emph{Tree Map}, and \emph{Carto Plot}. High-D also supports
clustering with k-means++ algorithm. High-D supports brushing and
linking, so all views are highly interactive and tightly linked. It is
possible to obtain specific numeric values of attributes by examining
the objects within the views, and dynamic queries on a combination of
attributes can be performed using range sliders. All actions are
accompanied by visual feedback within a common frame of reference. The
visual representations make it easy to spot unusual data points,
patterns, or anomalies.




\section{GGobi}

GGobi \parencite{cook2007interactive} is a visualization program for
exploring high-dimensional data. GGobi is written in C and is
available as a free and open-source software with the last update
issued on \yearmonthday{2012}{6}{10}. GGobi is available for Windows, Apple macOS,
and Linux.

GGobi can visualize custom datasets and provides a dynamic and
interactive graphics as tours, where data is displayed in an
animation. The data is also available in the following views:
\emph{Scatter Plot}, \emph{Scatter Plot Matrix}, \emph{Parallel
Coordinates}, \emph{Time Series}, \emph{Distributions} as bar
charts. GGobi supports limited brushing. Views offer limited
interactivity and interpretability and are not closely connected.


\section{mVis}

mVis \parencite{CHEGINI20199} is a visual analytics tool for
visualizing multi-dimensional data. mVis is written in Java and is
available as a free and open-source software with the last update
issued on \yearmonthday{2021}{1}{20}. mVis is available for Windows, 
macOS, and Linux.

mVis can visualize custom datasets and provides an overview of global
relationships between objects by using multiple views to show
different aspects of the data at the same time. mVis consists of four
data visualization views: \emph{Scatter Plot Matrix}, \emph{Scatter
Plot}, \emph{Similarity Map} as high dimensional projections (t-SNE,
PCA, and MDS), and \emph{Parallel Coordinates}. mVis also consists of
a panel for controlling data partitions. All of the visualizations are
interconnected through standard brushing and linking, so that any
changes or selections made in one view are reflected in all of the
other views. Additionally, the user has the ability to close,
rearrange, or expand any view as needed.

mVis also supports creating and modifying ML models, specifically
introducing an interactive visual labeling technique that allows an
analyst to build and iteratively improve a ML classification model for
multi-dimensional data sets. This technique combines linked
visualizations, clustering, and active learning to allow the analyst
to interactively label a multi-dimensional dataset in an efficient
manner.



\section{Improvise}

Improvise \parencite{Improvise} is a program that allows users to
create and interact with visualizations that are linked together in
various ways. Improvise is written in Java and is available as a free
and open-source software with the last update issued on 
\yearmonthday{2020}{10}{28}. Improvise is available for Windows, 
macOS, and Linux.

The program uses a shared-object coordination model and a declarative
visual query language to give users control over how data is displayed
in multiple views. This allows users to create visualizations with a
variety of coordination patterns, such as synchronized scrolling,
overview and detail, drill-down, and semantic zoom. Improvise also has
a user interface that allows users to build and explore visualizations
in a live environment, making it easy to modify visualizations as
needed. The goal of Improvise is to provide a high level of
coordination flexibility while also being easy to use. It is designed
to make it simple to create basic coordination patterns and also
possible to create more complex ones.




\section{MyBrush}

MyBrush \parencite{koytek2017mybrush} is an application that allows
users to customize and control the brushing and linking process in
their visualizations. It provides flexibility by allowing users to
specify the source, link, and target of multiple brushes, and supports
a variety of visualization types and multiple simultaneous
brushes. Improvise is written in JavaScript and is available as a free
and open-source web application with the last update issued on
\yearmonthday{2017}{9}{22}.

MyBrush serves as experimental software and offers limited
functionality. Its purpose is to explore the implemented brushing and
linking functionality.  A user can explore a predetermined set of data
with the following views: \emph{Scatter Plot}, \emph{Parallel
Coordinates}, and \emph{Bar Plot}. Any changes or selections made in
one of the visualizations will be reflected in all of the other views
because they are all interconnected through standard brushing and
linking.




\section{XDAT}

XDAT \parencite{XDAT} is a multidimensional data analysis tool
designed to help users quickly and easily extract valuable insights
from large, complex data sets with many variables. XDAT is written in
Java and is available as a free software with the last update issued
on \yearmonthday{2020}{8}{26}. XDAT is available for Windows, 
macOS, and Linux.

XDAT can visualize custom datasets and displays data in separate
views. XDAT displays the data with the following views: \emph{Parallel
Coordinates}, \emph{Table View} and \emph{Scatter Plot}. All of the
visualizations are interconnected through standard brushing and
linking, so that any changes or selections made in one view are
reflected in all of the other views.



\section{TabuVis}

TabuVis \parencite{nguyen2013tabuvis} is a flexible and customizable
visual analytics system that is optimized for analyzing
multidimensional data. Its visualizations can be customized by domain
experts to suit the specific needs of the data being analyzed. TabuVis
is written in Java and is available as a free software with the latest
update issued on \yearmonthday{2022}{2}{19}. TabuVis is available for 
Windows, macOS, and Linux.

TabuVis can visualize custom datasets and displays data in separate
views. TabuVis includes various features for analyzing data, such as
the ability to process data, add automatic marks, create custom
interactive visualizations, and filter the data. These features are
designed to support the entire data analysis process. TabuVis displays
the data in the following views: \emph{Scatter Plots}, \emph{Parallel
Coordinates}, and \emph{Star Plot}. TabuVis doesn't support brushing
and linking.




\section{Comparison of Tools}

A comparison of general information of MVA software can be seen in
Table~\ref{tab:SoftwareGeneral}. A comparison of features of MVA
software can be seen in Table~\ref{tab:SoftwareFeatures}.



\begin{table}[tp]
\tablestretch
\rowcolors{2}{}{tablerowcolour}
\centering
\begin{tabularx}{\linewidth}
{>{\kern-\tabcolsep}lllXl<{\kern-\tabcolsep}}
\toprule
\textbf{Software} & \textbf{Last update} & \textbf{Licence} & \textbf{Systems} & \textbf{Language} \\
\midrule
InfoScope & \yearmonthday{2007}{2}{9} & Free demo & Win, macOS, Linux & ? \\
%
High-D & \yearmonthday{2022}{12}{5} & Commercial & Windows, Apple macOS, and Linux & ? \\
%
GGobi & \yearmonthday{2012}{6}{10} & Free and open-source & Windows, Apple macOS, and Linux & C \\
%
mVis & \yearmonthday{2021}{1}{20} & Free and open-source & Windows, Apple macOS, and Linux & Java \\
%
Improvise & \yearmonthday{2020}{10}{28} & Free and open-source & Windows, Apple macOS, and Linux & Java \\
%
MyBrush & \yearmonthday{2017}{9}{22} & Free and open-source & Web Browser & JavaScript \\
%
XDAT & \yearmonthday{2020}{8}{26} & Free & Windows, Apple macOS, and Linux & Java \\
%
TabuVis & \yearmonthday{2022}{2}{19} & Free & Windows, Apple macOS, and Linux & Java \\
\bottomrule
\end{tabularx}

\caption[Overview of MVA Tools]
{%
Overview of MVA tools.
}
\label{tab:SoftwareGeneral}
\end{table}






\afterpage{%
\clearpage% Flush earlier floats (otherwise order might not be correct)
\thispagestyle{empty}% empty page style (?)
\begin{landscape}% Landscape page
\tablestretch
\rowcolors{2}{}{tablerowcolour}
\centering
\begin{tabularx}{\linewidth}
{>{\kern-\tabcolsep}lcccccccccccccc<{\kern-\tabcolsep}}
\toprule
\thead{Software} & \thead{Custom \\ Datasets} & \thead{Brushing} & \thead{Linking} & \thead{Table \\ View} & \thead{Scatter \\ Plot} & \thead{Scatter \\ Plot \\ Matrix} & \thead{Parallel \\ Coordinates} & \thead{Parallel \\ Coordinates \\ Matrix} & \thead{Similarity \\ Map} & \thead{Time \\ Series} & \thead{Distributions} & \thead{Table \\ Plot} & \thead{Tree \\ Map} & \thead{Carto \\ Plot} \\
\midrule
InfoScope & \checkmark & \checkmark & \checkmark & \checkmark & & & \checkmark & & \checkmark & & & & & \checkmark \\
%
High-D & \checkmark & \checkmark & \checkmark & \checkmark & \checkmark & \checkmark & \checkmark & \checkmark & \checkmark &  & \checkmark & \checkmark & \checkmark & \checkmark \\
%
GGobi & \checkmark & & & & \checkmark & \checkmark & \checkmark & & & \checkmark & \checkmark & & & \\
%
mVis & \checkmark & \checkmark & \checkmark & & \checkmark & \checkmark & \checkmark & & \checkmark & & & & & \\
%
Improvise & \checkmark & & & \checkmark & \checkmark & \checkmark & & & \checkmark & \checkmark & \checkmark & \checkmark & \checkmark & \checkmark \\
%
MyBrush & & \checkmark & \checkmark & & \checkmark & & \checkmark & & & & \checkmark & & & \\
%
XDAT & \checkmark & \checkmark & \checkmark & \checkmark & \checkmark & & \checkmark & & & & & & & \\
%
TabuVis & \checkmark & & & & \checkmark & & \checkmark & & & & & & & \\
\bottomrule
\end{tabularx}
\captionof{table}{Comparison of MVA Tools.}
\label{tab:SoftwareFeatures}
\end{landscape}
\clearpage% Flush page
}




\cleardoublepage
%----------------------------------------------------------------
%
%  File    :  survey-concl.tex
%
%  Author  :  Keith Andrews, IICM, TU Graz, Austria
% 
%  Created :  27 May 1993
% 
%  Changed :  16 Nov 2010
% 
%----------------------------------------------------------------


\chapter{Concluding Remarks}

\label{chap:Concl}



At the end of your survey, give a clear recommendation
as to which approach or tool to use in which situation.





\cleardoublepage
% for now, switch to language english
% hack to force unix date for biblio, biblatex 3.11
\begin{otherlanguage}{english}
\printbibliography[heading=bibintoc]
\end{otherlanguage}


\end{document}

